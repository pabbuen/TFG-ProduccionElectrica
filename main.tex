\documentclass[a4paper,12pt]{article}

\usepackage[spanish]{babel}     % Idioma del documento
\usepackage[nocheck]{fancyhdr}  % Encabezados y pies de página
\usepackage{geometry}           % Configuración de márgenes
\usepackage{setspace}           % Definir espacio interlineal
\usepackage{hyperref}           % hipervinculos
\usepackage{graphicx}           % Figuras e imagenes
\usepackage{amsmath}            % Ecuaciones matemáticas
\usepackage{amsfonts}           % Simbolos matemáticos
\usepackage{hyperref}
\usepackage{csquotes}

\usepackage[backend=biber,style=numeric,sorting=none]{biblatex}    % Referencias
\addbibresource{referencias.bib}    % Nombre del archivo .bib

% Configuración de márgenes
\geometry{
    a4paper,        % Tamaño del papel
    top=1.5cm,      % Margen superior
    bottom=1.5cm,   % Margen infecior
    inner=1.5cm,    % MI Páginas impares MD páginas pares
    outer=2cm,      % MD Páginas impares MI páginas pares
}

% Justificación completa, Espacio interlineal y Sangrado Parrafo
\setlength{\parskip}{0.5cm} % Sangrado de parrafo
\sloppy                     % Justificación
\setstretch{1.2}            % Espacio Interlineal

% Configuración Estilo de Página
\pagestyle{fancy}           % Estilo de la página
\fancyhf{}                  % Limpiar encabezado y pie de página
\fancyhead[L]{\leftmark}    % Titulo de la sección
\fancyfoot[R]{\thepage}     % Número de página

\begin{document}

\begin{titlepage}
    \centering
    {\bfseries\Large Universidad de Valladolid \par}
    \vspace{2cm}
    {\scshape\Huge Escuela de Ingeniería Informática \par}
    {\bfseries\Large Trabajo de Fin de Grado \par}
    \vspace{2cm}
    {\Large Grado en Ingeniería Informáctica 
    \\ (Mención de Computación) \par}
    \vspace{2.5cm}
    {\bfseries\Huge Implementación de algoritmos de aprendizaje automático para la
    predicción del consumo eléctrico en el sector energético \par}
    \vspace{2cm}
    \raggedleft
    {Autor: \\
    \bfseries D.Pablo Bueno Sánchez}
\end{titlepage}

\tableofcontents

\newpage

\section*{Agradecimientos}
\addcontentsline{toc}{section}{Agradecimientos}

\newpage

\section*{Resumen}
\addcontentsline{toc}{section}{Resumen}

\newpage

\section*{Abstract}
\addcontentsline{toc}{section}{Abstract}

\section*{Índice de Figuras}
\addcontentsline{toc}{section}{Índice de Figuras}

\newpage

\section*{Índice de Tablas}
\addcontentsline{toc}{section}{Índice de Tablas}

\newpage

\section*{Índice de Ecuaciones}
\addcontentsline{toc}{section}{Índice de Ecuaciones}



\section{Introducción}

\subsection{Contexto y Motivación}

\subsection{Objetivos}

\subsection{Organización y Estructura de la memoria}



\section{Estado del Arte}

\subsection{Introducción}

\subsection{Inteligencia Artificial}

\subsection{Machine Learning}

Dentro de la computación, los algoritmos son los encargados
de regir el compotamiento de las máquinas para la resolución
de problemas. Estos algoritmos reciben una entrada y, a partir
de una serie de pasos, producen una salida deseada para un 
problema en concreto. Sin embargo, existen problemas para los 
cuales no se tiene ningún algoritmo, aquí es donde es posible 
hacer uso del machine learning.

El Machine Learning es una rama dentro de la Inteligencia
Artificial la cual se encarga de permitir a las máquinas 
extraer, a partir de una gran cantidad de datos, una 
aproximación fiable de algoritmos para la resolución de 
diversos problemas.

\subsubsection{aprendizaje Supervisado}

% Definición Aprendizaje Supervisado [2]

El aprendizaje supervisado es una categoría del machine
learning caracterizada por el uso de datos etiquetados 
para el entrenamiento de los modelos.

El uso de conjuntos de datos etiquetados permite a los
modelos modificar gradualmente su comportamiento para
poder ajusterse al modelo final.

Con el fin de poder modificar los componentes del modelo
se utilizan funciones de pérdida, las cuales permiten
calcular la precisión de la salida del modelo con respecto
a la salida deseada, permitiendo así minimizar el error
hasta un mínimo deseable.

% Los modelos descritos abajo pasarlos a otras secciones
\subsubsection{Regresión Lineal}

\subsubsection{Árboles de Decisión}

\subsubsection{XGBoost}

\subsubsection{Otros Modelos}


\subsection{Deep Learning}

% Los modelos vistos abajo pasarlos a otras secciones
\subsubsection{Redes Neuronales Artificiales}   

\subsubsection{Redes Neuronales Recurrentes (RNN)}

Las redes neuronales recurrentes o RNN son un tipo de redes neuronales
artificiales diseñadas principalmente para la detección de patrones
en secuencias de datos \cite{RNN:desc}.

% SI ES POSIBLE HACER EL DIAGRAMA CON TIKZ 
\begin{figure}[h]
    \centering
    \includegraphics[width=0.8\textwidth]{
        referencias/imagenes/diagrama_RNN.png
    }
    \caption{Estructura de una red RNN con función 
    de activación ReLU}
    \label{fig:RNN}
\end{figure}

Al igual que otros tipos de redes neuronales, como los perceptrones
multicapa (MLP), la información se transmite desde la capa de entrada
$\mathbf{X}_{t}$ hasta la capa de salida $\mathbf{O}_{t}$ pasando
por el bloque de capa oculta $\mathbf{H}_{t}$. Sin embargo, 
a diferencia de estas, la entrada del bloque de la capa oculta
depende tanto de la entrada $\mathbf{X}_{t}$ como de la salida 
del bloque en su iteración anterior $\mathbf{H}_{t-1}$. Ver figura
\ref{fig:RNN}.

La salida de los bloques $\mathbf{H}_{t}$ y $\mathbf{O}_{t}$ viene
dada por las fórmulas:

\begin{equation}
    \mathbf{H}_{t} = \phi_{h}( \mathbf{X}_{t}\mathbf{W}_{xh} 
    + \mathbf{H}_{t-1}\mathbf{W}_{hh}+\mathbf{b}_{h} ) 
    \label{eq:rnn_ht}
\end{equation}
\begin{equation}
    \label{eq:rnn_ot}
    \mathbf{O}_{t} = \phi_{o} ( \mathbf{H}_{t} \mathbf{W}_{ho}
    + \mathbf{b}_{o})
\end{equation}

Donde, $\mathbf{X}_{t}$ es la salida de la capa de entrada,
$\mathbf{H}_{t}$ y $\mathbf{H}_{t-1}$ son las salidas de la capa 
oculta en los instantes t y t-1, $\mathbf{W}_{xh}$, $\mathbf{W}_{hh}$,
y $\mathbf{W}_{ho}$ son las matrices de pesos entre las diferentes
capas de la neurona y $\mathbf{b}_{h}$ y $\mathbf{b}_{o}$ son los
vectores de bias o sesgo para la capa oculta y la de salida
respectivamente. 

$\phi_{h}$ y $\phi_{o}$ son las funciones de activación 
de las capas oculta y de salida respectivamente. En el caso
de $\phi_{h}$ suele ser la función rectificador (ReLU) o 
tangente hiperbólica ($\tanh$).

\subsubsection{Backpropagation Through Time y problemas 
de evanescencia / explosión del gradiente}

Para poder comprender el principal problema de las redes 
neuronales recurrentes es necesário conocer el funcionamiento
del algoritmo de retropropagación a traves del tiempo o 
BPTT (BackPropagation Through Time).

Con el fin de entrenar una RNN, primero se debe establecer 
una función de pérdida $\mathfrak{L}$, la cual nos permite 
conocer la distancia entre el valor de salida de la neurona 
$\mathbf{O}_{t}$ y el valor real $\mathbf{Y}_{t}$.

\begin{equation}
    \mathfrak{L}(\mathbf{O}, \mathbf{Y}) = 
    \sum_{t=1}^{T}\ell_{t}(\mathbf{O}_{t}, \mathbf{Y}_{t})
    \label{eq:rnn_loss}
\end{equation}

Con el fin de minimizar dicha distancia, se han de calcular 
las derivadas parciales de $\mathfrak{L}$ con respecto 
los pesos de la neurona, $\mathbf{W}_{xh}$, $\mathbf{W}_{hh}$, 
y $\mathbf{W}_{ho}$. A partir de la ecuación \ref{eq:rnn_ot}
aplicando la regla de la cadena, se puede obtener:

\begin{equation}
    \frac{\partial\mathfrak{L}}{\partial\mathbf{W}_{ho}} = 
    \sum_{t=1}^{T}\frac{\partial\ell_{t}}{\partial\mathbf{O}_{t}}
    \cdot
    \frac{\partial\mathbf{O}_{t}}{\partial\phi_{o}}
    \cdot
    \frac{\partial\phi_{o}}{\partial\mathbf{W}_{ho}} = 
    \sum_{t=1}^{T}\frac{\partial\ell_{t}}{\partial\mathbf{O}_{t}}
    \cdot
    \frac{\partial\mathbf{O}_{t}}{\partial\phi_{o}}
    \cdot
    \mathbf{H}_{t}
    \label{eq:RNN_grads_Who}
\end{equation}
\begin{equation}
    \frac{\partial\mathfrak{L}}{\partial\mathbf{W}_{hh}} = 
    \sum_{t=1}^{T}\frac{\partial\ell_{t}}{\partial\mathbf{O}_{t}}
    \cdot
    \frac{\partial\mathbf{O}_{t}}{\partial\phi_{o}}
    \cdot
    \frac{\partial\phi_{o}}{\partial\mathbf{H}_{t}}
    \cdot
    \frac{\partial\mathbf{H}_{t}}{\partial\phi_{h}}
    \cdot
    \frac{\partial\phi_{h}}{\partial\mathbf{W}_{hh}} = 
    \sum_{t=1}^{T}\frac{\partial\ell_{t}}{\partial\mathbf{O}_{t}}
    \cdot
    \frac{\partial\mathbf{O}_{t}}{\partial\phi_{o}}
    \cdot
    \mathbf{W}_{ho}
    \cdot
    \frac{\partial\mathbf{H}_{t}}{\partial\phi_{h}}
    \cdot
    \frac{\phi_{h}}{\partial\mathbf{W}_{hh}}
    \label{eq:RNN_grads_Whh}
\end{equation}
\begin{equation}
    \frac{\partial\mathfrak{L}}{\partial\mathbf{W}_{hh}} = 
    \sum_{t=1}^{T}\frac{\partial\ell_{t}}{\partial\mathbf{O}_{t}}
    \cdot
    \frac{\partial\mathbf{O}_{t}}{\partial\phi_{o}}
    \cdot
    \frac{\partial\phi_{o}}{\partial\mathbf{H}_{t}}
    \cdot
    \frac{\partial\mathbf{H}_{t}}{\partial\phi_{h}}
    \cdot
    \frac{\partial\phi_{h}}{\partial\mathbf{W}_{xh}} = 
    \sum_{t=1}^{T}\frac{\partial\ell_{t}}{\partial\mathbf{O}_{t}}
    \cdot
    \frac{\partial\mathbf{O}_{t}}{\partial\phi_{o}}
    \cdot
    \mathbf{W}_{ho}
    \cdot
    \frac{\partial\mathbf{H}_{t}}{\partial\phi_{h}}
    \cdot
    \frac{\phi_{h}}{\partial\mathbf{W}_{xh}}
    \label{eq:RNN_grads_Wxh}
\end{equation}

Las ecuaciones \ref{eq:RNN_grads_Whh} y \ref{eq:RNN_grads_Wxh},
al depender $\mathbf{H}_{t}$ de sus valores anteriores, pueden
reescribirse como:
\begin{equation}
    \frac{\partial\mathfrak{L}}{\partial\mathbf{W}_{hh}} = 
    \sum_{t=1}^{T}\frac{\partial\ell_{t}}{\partial\mathbf{O}_{t}}
    \cdot
    \frac{\partial\mathbf{O}_{t}}{\partial\phi_{o}}
    \cdot
    \mathbf{W}_{ho}
    \sum_{k=1}^{t}
    \frac{\partial\mathbf{H}_{t}}{\partial\mathbf{H}_{k}}
    \cdot
    \frac{\partial\mathbf{H}_{k}}{\partial\mathbf{W}_{hh}}
    \label{eq:RNN_gradsTD_Whh}
\end{equation}
\begin{equation}
    \frac{\partial\mathfrak{L}}{\partial\mathbf{W}_{hh}} = 
    \sum_{t=1}^{T}\frac{\partial\ell_{t}}{\partial\mathbf{O}_{t}}
    \cdot
    \frac{\partial\mathbf{O}_{t}}{\partial\phi_{o}}
    \cdot
    \mathbf{W}_{ho}
    \sum_{k=1}^{t}
    \frac{\partial\mathbf{H}_{t}}{\partial\mathbf{H}_{k}}
    \cdot
    \frac{\partial\mathbf{H}_{k}}{\partial\mathbf{W}_{xh}}
    \label{eq:RNN_gradsTD_Wxh}
\end{equation}

Para entender el problema de la evanescencia y explosión 
del gradiente en las RNN es necesário fijarse en la expresión
$\frac{\partial\mathbf{H}_{t}}{\partial\mathbf{H}_{k}}$
de las ecuaciones \ref{eq:RNN_gradsTD_Whh} y 
\ref{eq:RNN_gradsTD_Wxh}.

El problema de la evanescencia del gradiente se produce 
cuando, en secuencias largas de datos, se establece unos
pesos iniciales menores a 1. A medida que se entrena 
la red RNN. Los gradientes se hacen menores con cada 
iteración hasta llegar a desaparecer, provocando que 
la red no pueda seguir entrenandose.

Con el problema de la explosión del gradiente se da 
una situación parecida a la evanescencia del gradiente,
pero al contrario que esta, al establecer los pesos 
iniciales a un valor superior a 1, los gradientes 
se irán haciendo mayores con cada iteración hasta que 
sean tan grandes que sea imposible entrenar la red 
debido a que el valor de los pesos sufren grandes cambios
en cada actualización, impidiendo encontrar un resultado 
óptimo.

\subsubsection{Redes Long Short-Term Memory (LSTM)}

Con el fin de solventar los problemas de la evanescencia
y explosión del gradiente de las redes recurrentes, 
en 1997 se ideo una nueva estructura de red recurrente.

La principal diferencia entre una red LSTM y una RNN 
es la presencia de un nuevo tipo de memoria a largo
plazo la cual ayuda a regular la memoria a corto 
plazo y predecir la salida de dicha red.

La estructura principal se caracteriza por su división 
en 3 puertas principales.

La primera de las puertas se denomina ``puerta del olvido''
($\mathbf{F}_{t}$). Es la parte de la red que se encarga 
de decidir que porcentaje de la memoria a largo plazo
($\mathbf{C}_{t-1}$) se debe olvidar o mantener. 

La segunda puerta es denomineada la ``puerta de entrada''
($\mathbf{I}_{t}$), determina la cantidad de memoria 
potencial a largo plazo debe añadirse.

Por último se encuentra la ``puerta de salida''
($\mathbf{O}_{t}$) la cual se encarga de determinar 
cuánta memoria a corto plazo debe recordarse.

\begin{equation}
    \mathbf{F}_{t} = \sigma\left(
        \mathbf{X}_{t}\mathbf{W}_{xf} +
        \mathbf{H}_{t-1}\mathbf{W}_{hf} + 
        \mathbf{b}_{f}
    \right)
    \label{eq:LSTM_Ft}
\end{equation}
\begin{equation}
    \mathbf{I}_{t} = \sigma\left(
        \mathbf{X}_{t}\mathbf{W}_{xi} +
        \mathbf{H}_{t-1}\mathbf{W}_{hi} + 
        \mathbf{b}_{i}
    \right)
    \label{eq:LSTM_It}
\end{equation}
\begin{equation}
    \mathbf{O}_{t} = \sigma\left(
        \mathbf{X}_{t}\mathbf{W}_{xo} +
        \mathbf{H}_{t-1}\mathbf{W}_{ho} + 
        \mathbf{b}_{o}
    \right)
    \label{eq:LSTM_Ot}
\end{equation}

En las ecuaciones \ref{eq:LSTM_Ft}, \ref{eq:LSTM_It} y 
\ref{eq:LSTM_Ot} puede observarse que el funcionamiento
de las puertas es identico salvo por los pesos 
$\mathbf{W}_{lj}$ y los bias $\mathbf{b}_{k}$ donde 
$l$,$j$,$k$ $\in x,f,i,o,h$ entre la entrada, la capa
oculta y las diferentes puertas.

% Figura Estructura LSTM
\begin{figure}[h]
    \centering
    **Estructura LSTM
    \caption{Estructura de una red LSTM}
    \label{fig:LSTM}
\end{figure}

Con el fin de actualizar la memoria a largo plazo, primero
es necesário calcular la memoria potencial 
$\tilde{\mathbf{C}}$, la cual se calcula de manera
similar al resto de puertas salvo por la función de 
activación, la cual en lugar de ser la función sigmoide
$\sigma$ utiliza una función de tangente hiperbólica
$\tanh$ para caluclar su salida. Vease la ecuación
\ref{eq:LSTM_Cp}.

\begin{equation}
    \tilde{\mathbf{C}_{t}} = \tanh\left(
        \mathbf{X}_{t}\mathbf{W}_{xc} +
        \mathbf{H}_{t-1}\mathbf{W}_{hc} + 
        \mathbf{b}_{c}
    \right)
    \label{eq:LSTM_Cp}
\end{equation}

Con todas las puertas definidas, el valor de la memoria a 
largo plazo viene dado por la ecuación \ref{eq:LSTM_C}

\begin{equation}
    \mathbf{C}_{t} = \mathbf{F}_{t} \odot \mathbf{C}_{t-1}
    + \mathbf{I}_{t} \odot \tilde{\mathbf{C}_{t}}
    \label{eq:LSTM_C}
\end{equation}

Por último, con la memoria a largo plazo actualizada,
tanto el valor de la salida como el de la capa oculta 
actualizada, vendrá dado por la función \ref{eq:LSTM_Ht}
\begin{equation}
    \mathbf{H}_{t} = \mathbf{O}_{t} \odot \tanh(\mathbf{C}_{t})
\end{equation}

Pese a solentar los problemas de evanescencia del gradiente,
es posible que pueda producirse explosión del gradiente
si no se predefinen bien las componentes. Pese a ello 
las redes LSTM son, actualmente, uno de los modelos de 
deep learning con mejores resultados para datos secuenciales
como series temporales o reconocimiento del lenguage.

\subsubsection{Gated Recurrent Unit (GRU)}

\subsubsection{Redes Neuronales Convolucionales (CNN)}

\subsubsection{Autoregressive Integrated Moving Average (ARIMA)}

\subsection{Series Temporales}

% Definición series temporales [4]
Una serie temporal es una secuencia de datos obtenidos
a lo largo de diferentes momentos ordenados conológicamente.
Dichos datos pueden obtenerse y ordenarse de dos formas diferentes:

\begin{itemize}
    \item \textbf{Equidistante en el tiempo:} Los datos se 
    obtienen en intervalos iguales de tiempo, ya sea de forma
    diaria, mensual, anual u horaria.

    \item \textbf{No equidistante en el tiempo:} Los datos 
    se obtienen en intervalos desiguales de tiempo, como 
    por ejemplo la medición de los kilometros recorridos por 
    un vehículo cada vez que se realiza una revisión en el 
    taller.
\end{itemize}

Una de las cualidades intrinsecas de las series temporales es 
que los datos adyacentes suelen tener una dependencia entre 
ellos. Dicha dependencia entre observaciones provee de un 
gran interes práctico. Una de las areas de aplicación de 
dicha dependencia es la de predicción de futuros valores dentro 
de la serie.

\subsubsection{Tendencia}

\subsubsection{Estacionalidad}

\subsubsection{Residual}

\subsubsection{Series Estacionarias}

\subsubsection{Predicción}

asdasd

\section{Datos}

En este apartado se tratará de describir tanto la fuente como 
una descripción detallada de los datos que se han utilizado
para la realización de las diferentes pruebas que se han 
realizado para el trabajo actual.

\subsection{Obtención}

El trabajo se ha realizado sobre un dataset obtenido de 
una competición de Kaggle llamada 
\href{https://www.kaggle.com/competitions/predict-energy-behavior-of-prosumers/data}{
    \textit{``Enefit - Predict Energy Behavior of Prosumers"}}
realizada entre el 1 de Noviembre de 2023 y el 31 de 
Enero de 2024.

Para ayudar con el pre-procesamiento de los datos, se ha usado el dataset
\href{https://www.kaggle.com/datasets/michaelo/fabiendaniels-mapping-locations-and-county-codes}{
    \textit{``fabiendaniels-mapping-locations-and-county-codes''}}, el cual ayuda a 
asignar un condado a una longitud y latitud especificas.

\subsection{Descripción}

El conjunto de datos consta de varios archivos csv
con diversas variables tanto energéticas como 
meteorológicas. Dichos archivos fueron obtenidos 
para predecir tanto la producción como el consumo
de energía de diversas personas en Estonia con 
instalación de paneles solares en sus hogares.

A continuacíon se encuentra una descriciòn de 
cada archivo y todas sus variables:

\begin{enumerate}
    \item \textbf{train.csv:}
    \begin{itemize}
        \item \textbf{country:} 
        ID del pais.
        \item \textbf{is\_business:}
        Indica si el consumidor es una empresa o
        un individuo.
        \item \textbf{product\_type:}
        ID de un mapa de tipo de contrato.
        \item \textbf{target:} 
        cantidad de consumo o producción del 
        segmento relevante durante la hora. 
        \item \textbf{is\_consumption:}
        Indica si el elemento de la columna 
        `target' es de consumo o producción.
        \item \textbf{datetime:}
        Hora local de Estonia en EET (UTC+2)/
        EEST(UTC+3). Define el inicio del periodo
        de una hora en el cual se da el valore de 
        `target'.
        \item \textbf{data\_block\_id:}
        
        \item \textbf{row\_id:}
        Identificador único de la fila
        \item \textbf{prediction\_unit\_id:}
        Identificador único formado por la unión de
        las columnas `country', `is\_business' y 
        `product\_type'.
    \end{itemize}
    
    \item \textbf{gas\_prizes.csv:}
    \begin{itemize}
        \item \textbf{origin\_date:}
        Fecha en el que los precios diarios
        se hicieron disponibles.
        \item \textbf{forecast\_date:}
        Fecha en la que los precios previstos
        deberían ser relevantes
        \item \textbf{[lowest/highest]\_price\_per\_mwh:}
        El precio mínimo/máximo del gas natural en el 
        mercado diario ese dia de negociación.
        \item \textbf{data\_block\_id}
    \end{itemize}

    \item \textbf{client.csv:}
    \begin{itemize}
        \item \textbf{product\_type:}
        \item \textbf{country:}
        ID del código del pais.
        \item \textbf{eic\_count:}
        Número agregado de puntos de consumo
        (EICs - European Identifier Code).
        \item \textbf{installed\_capacity:}
        Capacidad, en KWh, de los paneles solares
        fotovoltaicos instalados.
        \item \textbf{is\_business:}
        \item \textbf{date:}
        \item \textbf{data\_block\_id:}
    \end{itemize}

    \item \textbf{electricity\_prices.csv:}
    \begin{itemize}
        \item \textbf{origin\_date:}
        \item \textbf{forecast\_date:}
        \item \textbf{euros\_per\_mwh:}
        Precio de la electricidad en MWh
        \item \textbf{data\_block\_id:}
    \end{itemize}

    \item \textbf{forecast\_weather.csv:}
    \begin{itemize}
        \item \textbf{[latitude/longitude]:}
        Coordenadas del pronóstico.
        \item \textbf{origin\_datetime:}
        Marca de tiempo de la generación del 
        pronóstico.
        \item \textbf{hours\_ahead:}
        Horas entre la generación del prónostico 
        y el propio pronóstico del tiempo. Cada 
        pronóstico cubre un total de 48 horas.
        \item \textbf{temperature:}
        Temperatura del aire, en grados Celsius, 
        a 2 metros sobre el nivel del suelo. Estimado
        al final del periodo de 1 hora.
        \item \textbf{dewpoint:}
        La temperatura del punto de rocío, en 
        grados celsius, a 2 metros sobre el nivel del 
        suelo. Estimado al final del periodo de 1 hora.
        \item \textbf{cloudcover\_[low/mid/high/total]:}
        Porcentaje de cielo cubierto en las siguientes 
        bandas de altitud: 0-2, 2-6, 6+ y total en km.
        Estimado al final del periodo de 1 hora.
        \item \textbf{10\_metre\_[u/v]\_wind\_component:}
        Componente de la velocidad del viento en dirección
        [este/norte] a 10 metros sobre la superficie, en 
        metros por segundo (m/s).
        Estimado al final del periodo de 1 hora.
        \item \textbf{data\_blcok\_id}
        \item \textbf{forecast\_datetime:}
        Marca de tiempo de la predicción del tiempo.
        Generado de la suma de `origin\_datetime' y 
        `hours\_ahead'. Representa el inicio del periodo
        de 1 hora donde se pronostican los datos del clima.
        \item \textbf{direct\_solar\_radiation:}
        \item \textbf{surface\_solar\_radiation\_downwards:}
        \item \textbf{snowfall:}
        \item \textbf{total\_precipitation:}
    \end{itemize}

    \item \textbf{historical\_weather.csv:}
    \begin{itemize}
        \item \textbf{datetime:}
        \item \textbf{temperature:}
        \item \textbf{dewpoint:}
        \item \textbf{rain:}
        \item \textbf{snowfall:}
        \item \textbf{surface\_pressure:}
        \item \textbf{cloudcover\_[low/mid/high/total]:}
        \item \textbf{windspeed\_10m:}
        \item \textbf{winddirection\_10m:}
        \item \textbf{shortwave\_radiation:}
        \item \textbf{direct\_solar\_radiation:}
        \item \textbf{diffuse\_radiation:}
        \item \textbf{[latitude/longitude]:}
        \item \textbf{data\_block\_id:}
    \end{itemize}

\end{enumerate}

\subsection{Preprocesamiento}
preprocesamiento

\section{Predicción del consumo energético}

\subsection{Base teórica}

\subsubsection{Algoritmos}

\subsubsection{Funciones de pérdida}

\subsubsection{Optimizadores}

\subsubsection{Modelo 1}

\subsubsection{Modelo 2}

\subsubsection{Modelo 3}

\section{Predicción de producción energética}

\subsection{Base teórica}

\subsubsection{Algoritmos}

\subsubsection{Funciones de pérdida}

\subsubsection{Optimizadores}

\subsection{Implementación}

\subsubsection{Preprocesamiento de Datos}

\subsubsection{Clasificación Modelo 1}

\subsubsection{Clasificación Modelo 2}

\subsubsection{Clasificación Modelo 3}

asdasd w fqf

\section{Conclusiones}
asdjsadjajsd

\subsection{Predicción de consumo energético}
akdjlaskdjkasj

\subsection{Predicción de producción energética}

\subsection{Comparación de resultados}

\subsection{Aplicabilidad de los algoritmos de predicción}

\section{Trabajo futuro}

\section*{Bibliografía}
\addcontentsline{toc}{section}{Bibliografía}

\printbibliography

\end{document}