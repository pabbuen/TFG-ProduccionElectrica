\documentclass[a4paper,12pt]{article}

\usepackage[spanish]{babel} % Idioma del documento
\usepackage{fancyhdr}       % Encabezados y pies de página
\usepackage{geometry}       % Configuración de márgenes
\usepackage{setspace}       % Definir espacio interlineal
\usepackage{hyperref}       % hipervinculos

% Configuración de márgenes
\geometry{
    a4paper,        % Tamaño del papel
    top=1.5cm,      % Margen superior
    bottom=1.5cm,   % Margen infecior
    inner=1.5cm,    % MI Páginas impares MD páginas pares
    outer=2cm,      % MD Páginas impares MI páginas pares
}

% Justificación completa, Espacio interlineal y Sangrado Parrafo
\setlength{\parskip}{0.5cm} % Sangrado de parrafo
\sloppy                     % Justificación
\setstretch{1.2}            % Espacio Interlineal

% Configuración Estilo de Página
\pagestyle{fancy}           % Estilo de la página
\fancyhf{}                  % Limpiar encabezado y pie de página
\fancyhead[L]{\leftmark}    % Titulo de la sección
\fancyfoot[R]{\thepage}     % Número de página

\begin{document}

\begin{titlepage}
    \centering
    {\bfseries\Large Universidad de Valladolid \par}
    \vspace{2cm}
    {\scshape\Huge Escuela de Ingeniería Informática \par}
    {\bfseries\Large Trabajo de Fin de Grado \par}
    \vspace{2cm}
    {\Large Grado en Ingeniería Informáctica 
    \\ (Mención de Computación) \par}
    \vspace{2.5cm}
    {\bfseries\Huge Implementación de algoritmos de aprendizaje automático para la
    predicción del consumo eléctrico en el sector energético \par}
    \vspace{2cm}
    \raggedleft
    {Autor: \\
    \bfseries D.Pablo Bueno Sánchez}
\end{titlepage}

\tableofcontents

\newpage

\section*{Agradecimientos}
\addcontentsline{toc}{section}{Agradecimientos}

\newpage

\section*{Resumen}
\addcontentsline{toc}{section}{Resumen}

\newpage

\section*{Abstract}
\addcontentsline{toc}{section}{Abstract}

\section*{Índice de Figuras}
\addcontentsline{toc}{section}{Índice de Figuras}

\newpage

\section*{Índice de Tablas}
\addcontentsline{toc}{section}{Índice de Tablas}

\newpage

\section*{Índice de Ecuaciones}
\addcontentsline{toc}{section}{Índice de Ecuaciones}



\section{Introducción}

\subsection{Contexto y Motivación}

\subsection{Objetivos}

\subsection{Organización y Estructura de la Memória}



\section{Estado del Arte}

\subsection{Introducción}

\subsection{Inteligencia Artificial}

\subsection{Machine Learning}

Dentro de la computación, los algoritmos son los encargados
de regir el compotamiento de las máquinas para la resolución
de problemas. Estos algoritmos reciben una entrada y, a partir
de una serie de pasos, producen una salida deseada para un 
problema en concreto. Sin embargo, existen problemas para los 
cuales no se tiene ningún algoritmo, aquí es donde es posible 
hacer uso del machine learning.

El Machine Learning es una rama dentro de la Inteligencia
Artificial la cual se encarga de permitir a las máquinas 
extraer, a partir de una gran cantidad de datos, una 
aproximación fiable de algoritmos para la resolución de 
diversos problemas.

\subsubsection{aprendizaje Supervisado}

% Definición Aprendizaje Supervisado [2]

El aprendizaje supervisado es una categoría del machine
learning caracterizada por el uso de datos etiquetados 
para el entrenamiento de los modelos.

El uso de conjuntos de datos etiquetados permite a los
modelos modificar gradualmente su comportamiento para
poder ajusterse al modelo final.

Con el fin de poder modificar los componentes del modelo
se utilizan funciones de pérdida, las cuales permiten
calcular la precisión de la salida del modelo con respecto
a la salida deseada, permitiendo así minimizar el error
hasta un mínimo deseable.

% Los modelos descritos abajo pasarlos a otras secciones
\subsubsection{Regresión Lineal}

\subsubsection{Árboles de Decisión}

\subsubsection{XGBoost}

\subsubsection{Otros Modelos}


\subsection{Deep Learning}

% Los modelos vistos abajo pasarlos a otras secciones
\subsubsection{Redes Neuronales Artificiales}

\subsubsection{Redes Neuronales Recurrentes (RNN)}

\subsubsection{Redes Long Short-Term Memory (LSTM)}

\subsubsection{Gated Recurrent Unit (GRU)}

\subsubsection{Redes Neuronales Convolucionales (CNN)}

\subsubsection{Autoregressive Integrated Moving Average (ARIMA)}


\subsection{Series Temporales}

Una serie temporal es una secuencia de datos obtenidos
a lo largo de diferentes momentos ordenados conológicamente.
Dichos datos pueden obtenerse y ordenarse de dos formas diferentes:

\begin{itemize}
    \item \textbf{Equidistante en el tiempo:} Los datos se 
    obtienen en intervalos iguales de tiempo, ya sea de forma
    diaria, mensual, anual u horaria.

    \item \textbf{No equidistante en el tiempo:} Los datos 
    se obtienen en intervalos desiguales de tiempo, como 
    por ejemplo la medición de los kilometros recorridos por 
    un vehículo cada vez que se realiza una revisión en el 
    taller.
\end{itemize}

% Definición series temporales [4]
Una de las cualidades intrinsecas de las series temporales es 
que los datos adyacentes suelen tener una dependencia entre 
ellos. Dicha dependencia entre observaciones provee de un 
gran interes práctico. Una de las areas de aplicación de 
dicha dependencia es la de predicción de futuros valores dentro 
de la serie.



\subsubsection{Objetivos}

\subsubsection{Tendencia}

\subsubsection{Estacionalidad}

\subsubsection{Residual}

\subsubsection{Series Estacionarias}

\subsubsection{Predicción}


\section{Datos}

\subsection{Obtención}

\subsection{Descripción}

\section{Descripción del consumo energético}

\subsection{Base teórica}

\subsubsection{Algoritmos}

\subsubsection{Funciones de pérdida}

\subsubsection{Optimizadores}

\subsubsection{Modelo 1}

\subsubsection{Modelo 2}

\subsubsection{Modelo 3}

\section{Predicción de producción energética}

\subsection{Base teórica}

\subsubsection{Algoritmos}

\subsubsection{Funciones de pérdida}

\subsubsection{Optimizadores}

\subsection{Implementación}

\subsubsection{Preprocesamiento de Datos}

\subsubsection{Clasificación Modelo 1}

\subsubsection{Clasificación Modelo 2}

\subsubsection{Clasificación Modelo 3}

\section{Conclusiones}

\subsection{Predicción de consumo energético}

\subsection{Predicción de producción energética}

\subsection{Comparación de resultados}

\subsection{Aplicabilidad de los algoritmos de predicción}

\section{Trabajo futuro}

\section*{Bibliografía}
\addcontentsline{toc}{section}{Bibliografía}

\end{document}