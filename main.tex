\documentclass[a4paper,12pt]{article}

\usepackage[spanish]{babel} % Idioma del documento
\usepackage{fancyhdr}       % Encabezados y pies de página
\usepackage{geometry}       % Configuración de márgenes
\usepackage{setspace}       % Definir espacio interlineal
\usepackage{hyperref}       % hipervinculos

% Configuración de márgenes
\geometry{
    a4paper,        % Tamaño del papel
    top=1.5cm,      % Margen superior
    bottom=1.5cm,   % Margen infecior
    inner=1.5cm,    % MI Páginas impares MD páginas pares
    outer=2cm,      % MD Páginas impares MI páginas pares
}

% Justificación completa, Espacio interlineal y Sangrado Parrafo
\setlength{\parskip}{0.5cm} % Sangrado de parrafo
\sloppy                     % Justificación
\setstretch{1.2}            % Espacio Interlineal

% Configuración Estilo de Página
\pagestyle{fancy}           % Estilo de la página
\fancyhf{}                  % Limpiar encabezado y pie de página
\fancyhead[L]{\leftmark}    % Titulo de la sección
\fancyfoot[R]{\thepage}     % Número de página

\begin{document}

\begin{titlepage}
    \centering
    {\bfseries\Large Universidad de Valladolid \par}
    \vspace{2cm}
    {\scshape\Huge Escuela de Ingeniería Informática \par}
    {\bfseries\Large Trabajo de Fin de Grado \par}
    \vspace{2cm}
    {\Large Grado en Ingeniería Informáctica 
    \\ (Mención de Computación) \par}
    \vspace{2.5cm}
    {\bfseries\Huge Implementación de algoritmos de aprendizaje automático para la
    predicción del consumo eléctrico en el sector energético \par}
    \vspace{2cm}
    \raggedleft
    {Autor: \\
    \bfseries D.Pablo Bueno Sánchez}
\end{titlepage}

\tableofcontents

\newpage

\section*{Agradecimientos}
\addcontentsline{toc}{section}{Agradecimientos}

\newpage

\section*{Resumen}
\addcontentsline{toc}{section}{Resumen}

\newpage

\section*{Abstract}
\addcontentsline{toc}{section}{Abstract}

\section*{Índice de Figuras}
\addcontentsline{toc}{section}{Índice de Figuras}

\newpage

\section*{Índice de Tablas}
\addcontentsline{toc}{section}{Índice de Tablas}

\newpage

\section*{Índice de Ecuaciones}
\addcontentsline{toc}{section}{Índice de Ecuaciones}

\section{Introducción}

\subsection{Contexto y Motivación}

\subsection{Objetivos}

\subsection{Organización y Estructura de la Memória}

\section{Estado del Arte}

\subsection{Introducción}

\subsection{Historia}

\subsection{Machine Learning}

\subsubsection{aprendizaje Supervisado}

\subsubsection{Regresión Lineal}

\subsubsection{Árboles de Decisión}

\subsubsection{XGBoost}

\subsubsection{Otros Modelos}

\subsection{Deep Learning}

\subsubsection{Redes Neuronales Artificiales}

\subsubsection{Redes Neuronales Recurrentes (RNN)}

\subsubsection{Redes Long Short-Term Memory (LSTM)}

\subsubsection{Gated Recurrent Unit (GRU)}

\subsubsection{Redes Neuronales Convolucionales (CNN)}

\subsubsection{Autoregressive Integrated Moving Average (ARIMA)}

\subsection{Series Temporales}

\subsubsection{Objetivos}

\subsubsection{Tendencia}

\subsubsection{Estacionalidad}

\subsubsection{Predicción}

\section{Datos}

\subsection{Obtención}

\subsection{Descripción}

\section{Descripción del consumo energético}

\subsection{Base teórica}

\subsubsection{Algoritmos}

\subsubsection{Funciones de pérdida}

\subsubsection{Optimizadores}

\subsubsection{Modelo 1}

\subsubsection{Modelo 2}

\subsubsection{Modelo 3}

\section{Predicción de producción energética}

\subsection{Base teórica}

\subsubsection{Algoritmos}

\subsubsection{Funciones de pérdida}

\subsubsection{Optimizadores}

\subsection{Implementación}

\subsubsection{Preprocesamiento de Datos}

\subsubsection{Clasificación Modelo 1}

\subsubsection{Clasificación Modelo 2}

\subsubsection{Clasificación Modelo 3}

\section{Conclusiones}

\subsection{Predicción de consumo energético}

\subsection{Predicción de producción energética}

\subsection{Comparación de resultados}

\subsection{Aplicabilidad de los algoritmos de predicción}

\section{Trabajo futuro}

\section*{Bibliografía}
\addcontentsline{toc}{section}{Bibliografía}

\end{document}